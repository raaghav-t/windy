\documentclass[11pt]{article}


% ------------------------------------------------
% Packages
% ------------------------------------------------
\usepackage{amsmath, amssymb, amsthm}
\usepackage{graphicx}
\usepackage{cite}
\usepackage{hyperref}
\usepackage{enumitem}
\usepackage[left=0.9in,right=0.9in,top=1in,bottom=1in]{geometry}

\hypersetup{
    colorlinks=true,
    linkcolor=blue,
    citecolor=blue,
    urlcolor=blue
}

% ------------------------------------------------
% Theorem Environments
% ------------------------------------------------
\newtheorem{theorem}{Theorem}
\newtheorem{lemma}{Lemma}
\newtheorem{proposition}{Proposition}
\newtheorem{corollary}{Corollary}
\newtheorem{remark}{Remark}

% ------------------------------------------------
% Title (Keep it clear, not fancy)
% ------------------------------------------------
\title{Proposal: Reinforcement Learning for xGame}
\author{Melissa Osheroff, Tien Nguyen, Raaghav Thirumaligai}
\date{\today}

\begin{document}

\maketitle

% ------------------------------------------------
% Abstract
% ------------------------------------------------
\begin{abstract}
Brief description of the problem addressed.

Statement of the main result.

Key conclusion or implication.

Optional mention of application or numerical illustration.

% Avoid citations, equations, and references here.
\end{abstract}

% =================================================
\section{Introduction}
% =================================================

% Paragraph 1: Problem description
Brief description of the problem and context.

% Paragraph 2: Literature review (short version)
Discussion of related work and positioning within the literature.

% Paragraph 3: Contribution
The main contribution of this paper is:
\begin{itemize}
    \item Contribution 1
    \item Contribution 2
\end{itemize}

% Final paragraph: Organization
The remainder of this paper is organized as follows.
Section~\ref{sec:related} discusses related work.
Section~\ref{sec:problem} formulates the problem.
Section~\ref{sec:main} presents the main result.
Section~\ref{sec:discussion} discusses consequences.
Section~\ref{sec:proofs} provides proofs.
Section~\ref{sec:examples} presents applications.
Section~\ref{sec:conclusion} concludes the paper.

\paragraph*{Notation}
Define notation only if widely used. Otherwise introduce symbols when needed.

% =================================================
\section{Related Work}
\label{sec:related}
% =================================================

Organize into paragraphs by topic or chronology.

Be fair and avoid overselling your contribution.

% =================================================
\section{Problem Statement}
\label{sec:problem}
% =================================================

\subsection*{System Description}
Describe the process/system and define state variables.

\subsection*{Actuation Mechanism}
Describe input/control variables and dynamics:
\begin{equation}
    \dot{x} = f(x,u,w)
\end{equation}

\subsection*{Sensing Mechanism}
Define outputs:
\begin{equation}
    y = h(x)
\end{equation}

\subsection*{Objective}
Design a controller/algorithm that achieves:
\begin{equation}
    \text{Minimize } J(x,u)
\end{equation}
subject to system constraints.

Include a figure here if helpful and refer to it explicitly.

% =================================================
\section{Main Result}
\label{sec:main}
% =================================================

State the main result clearly and early.

\begin{theorem}[Main Stability Result]
Consider the system defined in Section~\ref{sec:problem}.
Assume:
\begin{enumerate}
    \item Assumption 1
    \item Assumption 2
\end{enumerate}
Then, for all admissible initial conditions,
\begin{equation}
    \lim_{t \to \infty} x(t) = 0.
\end{equation}
\end{theorem}

\begin{remark}[Interpretation]
Provide insight into what the result means.
\end{remark}

Avoid cluttering this section with long proofs.

% =================================================
\section{Discussion}
\label{sec:discussion}
% =================================================

Discuss special cases, corollaries, or implications.

\begin{corollary}
State meaningful corollaries only.
\end{corollary}

% =================================================
\section{Proofs}
\label{sec:proofs}
% =================================================

\begin{proof}[Proof of Theorem 1]
Provide clear reasoning.
Avoid phrases like “Clearly” or “It is obvious”.
\end{proof}

% =================================================
\section{Applications and Examples}
\label{sec:examples}
% =================================================

Illustrate how the result is used in practice.

Include simulations only if they provide insight.

If randomness is involved:
\begin{itemize}
    \item Use Monte Carlo runs
    \item Report averages and confidence intervals
\end{itemize}

% =================================================
\section{Conclusions and Future Work}
\label{sec:conclusion}
% =================================================

% Paragraph 1: Summary
Summarize the main result and contribution.

% Paragraph 2: Future work
Discuss open problems and future directions.

% =================================================
\appendix
\section{Technical Lemmas}

Place secondary technical results here if needed.

% =================================================
\bibliographystyle{ieeetr}
\bibliography{references}

\end{document}
