\documentclass[11pt]{article}


% ------------------------------------------------
% Packages
% ------------------------------------------------
\usepackage{amsmath, amssymb, amsthm}
\usepackage{graphicx}
\usepackage{cite}
\usepackage{hyperref}
\usepackage{enumitem}
\usepackage[left=0.9in,right=0.9in,top=1in,bottom=1in]{geometry}

\hypersetup{
    colorlinks=true,
    linkcolor=blue,
    citecolor=blue,
    urlcolor=blue
}


\title{Dynammic Programming in Rook Endgames}
\author{Melissa Osheroff, Tien Nguyen, Raaghav Thirumaligai}
\date{\today}

\begin{document}

\maketitle

% ------------------------------------------------
% Abstract
% ------------------------------------------------
% ------------------------------------------------
% Abstract
% ------------------------------------------------
\begin{abstract}
    This project proposes the software implentation of a dynamic program that finds the best outcome for Player 1 in chess endgame. Player 1 has a rook and a king left, whereas Player 2 has only a king. Thus, this is a zero-sum game where Player 1 is trying to checkmate and Player 2 is trying to stalemate. 
\end{abstract}

% =================================================
\section{Introduction}
% =================================================

% Paragraph 1: Problem description
Brief description of the problem and context.

% Paragraph 2: Literature review (short version)
Discussion of related work and positioning within the literature.

% Paragraph 3: Contribution
The main contribution of this paper is:
\begin{itemize}
    \item Contribution 1
    \item Contribution 2
\end{itemize}

% Final paragraph: Organization
The remainder of this paper is organized as follows.
Section~\ref{sec:related} discusses related work.
Section~\ref{sec:problem} formulates the problem.
Section~\ref{sec:main} presents the main result.
Section~\ref{sec:discussion} discusses consequences.
Section~\ref{sec:proofs} provides proofs.
Section~\ref{sec:examples} presents applications.
Section~\ref{sec:conclusion} concludes the paper.

% =================================================
\section{Problem Statement}
\label{sec:problem}
% =================================================

\subsection*{System Description}
We want to find the most optimal path to winning a game of chess, given a rook(H1) and a king(A1) against a king(A8). The optimal path is defined as a checkmate occuring before stalemate for Player 1 and occurring with the least amount of moves possible.   

% =================================================
\section{Proposed Approach}
% =================================================

We propose the following steps:

\begin{enumerate}
    \item \textbf{Formal Modeling:} Define xGame in normal-form and, if appropriate, extensive-form representation. Identify strategy spaces and payoff functions.
    \item \textbf{Equilibrium Analysis:} Characterize pure and mixed Nash equilibria. Analyze existence, uniqueness, and structural properties.
    \item \textbf{Learning Dynamics:} Implement reinforcement learning agents (e.g., Q-learning or policy-gradient methods) and study convergence behavior under repeated play.
    \item \textbf{Comparative Analysis:} Compare learned strategies with theoretical equilibria and evaluate efficiency and stability.
\end{enumerate}


\bibliographystyle{ieeetr}
\bibliography{references}

\end{document}
